{\color{gray}\hrule}
\begin{center}
\section{Conclusions}
\bigskip
\end{center}
{\color{gray}\hrule}
\vspace{0.5cm}

\begin{multicols}{2}
Energy consumption optimization in container-based cloud environments has evolved from early heuristic methods to sophisticated, predictive, and container-aware scheduling approaches. Early consolidation techniques demonstrated significant energy savings through aggressive host shutdown and DVFS, but their reliance on static thresholds often limited adaptability under dynamic workloads\cite{carrega_energy-aware_2017}. In contrast, predictive algorithms that incorporate workload forecasting and dynamic resource allocation have achieved further energy reductions (typically 30--35\%) while balancing the trade-off between energy savings and service quality\cite{dabbagh_energy-efficient_2015,bui_energy_2017}.

The transition from VM-based to container-based solutions has enabled finer-grained resource management, with approaches such as flow-network optimization for container placement proving effective even in large-scale simulations\cite{hu_concurrent_2020}. Additionally, strategies like brownout scheduling\cite{xu_energy_2016} and meta-heuristic or AI-driven methods\cite{tan_hybrid_2019,shi_energy-aware_2018} have shown promise in achieving global optimization over complex infrastructures, albeit with increased computational overhead.

A key insight from the surveyed literature is that no single method outperforms all others under every scenario; instead, each approach excels in specific operational contexts. For instance, consolidation-focused strategies yield high energy savings but require careful management to avoid SLA violations, whereas QoS-aware methods maintain performance at the expense of some energy efficiency\cite{li_sla-aware_2018}. Moreover, the emerging integration of renewable energy awareness into scheduling algorithms\cite{kumar_renewable_2019} points toward a future where energy management is not only efficient but also environmentally sustainable.

Looking ahead, future research should aim to integrate these diverse techniques into hybrid frameworks that combine predictive scheduling, dynamic consolidation, and real-time QoS monitoring. This integration, alongside the continued evolution of container orchestration platforms, is essential for achieving the dual goals of operational efficiency and sustainable cloud computing.
\end{multicols}