Growing demand for cloud services has shifted data center design from performance-centric to energy-efficient architectures. Containerization, a lightweight alternative to VMs, enables flexible resource allocation and reduced overhead. This survey reviews energy optimization techniques in container-based cloud environments by systematically selecting studies from 2010–2020 and categorizing them by scheduling scope, optimization method, energy-saving strategy, and Docker maturity. Our analysis covers early VM consolidation methods that achieved up to 83 \% energy savings, predictive scheduling, and advanced container orchestration. Overall, container-based optimizations—particularly those leveraging AI and hybrid heuristics—significantly reduce energy consumption while maintaining Quality of Service (QoS), pointing to promising future research directions.

\textbf{Keywords} 
Cloud Computing; Containerization; Energy Optimization; Microservices; Resource Allocation
