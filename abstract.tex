Due to growing client demand and reliance on cloud-based services, the approach to the architecture of data centers and cloud infrastructure is constantly evolving to keep up with high demand. The primary concern used to be optimization for the sake of computing power; now, it is energy efficiency for economic and environmental reasons. The shift to a container-based approach has led to new possibilities for optimization and efficiency, mainly through new scheduling approaches for virtual machine consolidation and infrastructure improvements. \textcolor{blue}{You should add here information of what this paper is about. (Read some abstracts of state-of-the-art papers for reference)}
\textcolor{red}{I don't think that this part is necessary: $\rightarrow$ \textit{However, optimization is a balancing act between different performance metrics and efficiency, leading to various approaches that focus on optimizing different metrics. One of the key metrics that is crucial for data center clients is the Service Level Agreement Violation (SLAV) rate. A container-based approach with proper scheduling and workload distribution allows meeting client needs in terms of performance and responsiveness while reducing energy consumption.}}

\textbf{Keywords} 
    
Cloud Computing $\cdot$ Containerization $\cdot$ Energy Optimization $\cdot$ Microservices $\cdot$ Resource Allocation.
